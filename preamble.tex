%%% Local Variables:
%%% mode: latex
%%% TeX-master: "doc"
%%% coding: utf-8
%%% End:
% !TEX TS-program = pdflatexmk
% !TEX encoding = UTF-8 Unicode
% !TEX root = doc.tex

% Lorem Ipsum
\usepackage{lipsum}

% Language and Font
\usepackage[utf8]{inputenc}
\usepackage[T1]{fontenc}
\usepackage[ngerman, english]{babel}
\usepackage{lmodern}

% Mathematical Symbols
\usepackage{amssymb}

% Matrix etc.
\usepackage{amsmath}

% Images
\usepackage{graphicx} % for including graphics
\graphicspath{ {img/} } % root directory for graphics
\usepackage{subcaption} % figures inside figures (e.g. for side by side figures)
\usepackage{float} % required for positioning figures with [H] (and lots of other stuff)
\usepackage{wrapfig} % floating figures

% Import PDFs
\usepackage{pdfpages}


\usepackage[headsepline, footsepline, plainfootsepline]{scrlayer-scrpage}
\clearpairofpagestyles
\automark{chapter}
\ihead{\leftmark{}}
\ofoot[\thepage]{\thepage}

\setlength{\parindent}{0cm} % Kein Einrücken bei einem neuen Paragraphen

% Paragraph so fertigmachen, dass er als Titel angezeigt werden kann
\makeatletter
\renewcommand\paragraph{\@startsection{paragraph}{4}{\z@}%
            {-2.5ex\@plus -1ex \@minus -.25ex}%
            {1.25ex \@plus .25ex}%
            {\normalfont\normalsize\itshape\bfseries}}
\makeatother

% Package um Todo Notizen einzufügen und farblich hervorzuheben
\usepackage[ngerman,linecolor=gray,bordercolor=gray,backgroundcolor=yellow]{todonotes}

% Code Highlighting
\usepackage{listings}
\definecolor{mygreen}{rgb}{0,0.6,0}
\definecolor{mymauve}{rgb}{0.58,0,0.82}
\lstset{ %
  basicstyle=\ttfamily,        % size of fonts used for the code
  breaklines=true,                 % automatic line breaking only at whitespace
  captionpos=b,                    % sets the caption-position to bottom
  commentstyle=\color{mygreen},    % comment style
  escapeinside={\%*}{*)},          % if you want to add LaTeX within your code
  keywordstyle=\color{blue},       % keyword style
  stringstyle=\color{mymauve},     % string literal style
  extendedchars=true,
  frame=single,
  frameround=tttt,
  framexleftmargin=1pt,
  framexrightmargin=1pt
}

% Hyperlinks
\usepackage[hidelinks]{hyperref}
\hypersetup{%
  pdftitle={Level Of Detail Generierungs-System für 3D Webapplikationen},
% Use \plainauthor for final version.
  pdfauthor={Marc Berli und Simon Stucki},
%  pdfauthor={\emptyauthor},
  pdfdisplaydoctitle=true, % For Accessibility
  bookmarksnumbered,
  pdfstartview={FitH}
}

% Enumerated list modification
\usepackage{enumitem}

% Diagrams
\usepackage{pgfplots}
\usepackage{pgfplotstable}
\usepackage{sfmath}
\usetikzlibrary{patterns}

% Entfernt den Blocksatz beim Literaturverzeichnis und formatiert schöner als nur \raggedright
\usepackage{ragged2e}

\newcommand{\pro}{\item[$+$]}
\newcommand{\con}{\item[$-$]}

% shortcut for emphasize
\newcommand{\e}{\emph}

% Shortcut for first glossary reference: lower-case & singular
\newcommand{\fgls}[2]{{
  \newglossaryentry{#1}
  {
    name={#1},
    description={{#2}}
  }
  \gls{#1}
  \footnote{#2}
}}

% Shortcut for first glossary reference: lower-case & plural
\newcommand{\fglspl}[2]{{
  \newglossaryentry{#1}
  {
    name={#1},
    description={{#2}}
  }
  \glspl{#1}
  \footnote{#2}
}}

% Shortcut for first glossary reference: Upper-case & singular
\newcommand{\fGls}[2]{{
  \newglossaryentry{#1}
  {
    name={#1},
    description={{#2}}
  }
  \Gls{#1}
  \footnote{#2}
}}

% Shortcut for first glossary reference: Upper-case & plural
\newcommand{\fGlspl}[2]{{
  \newglossaryentry{#1}
  {
    name={#1},
    description={{#2}}
  }
  \Glspl{#1}
  \footnote{#2}
}}