%%% Local Variables:
%%% mode: latex
%%% TeX-master: "../doc"
%%% coding: utf-8
%%% End:
% !TEX TS-program = pdflatexmk
% !TEX encoding = UTF-8 Unicode
% !TEX root = ../doc.tex
\section{Vergleich LOD Systeme}
Die verschiedenen LOD Systeme sowie ihre Vor- und Nachteile werden in \autoref{chap:differentLodApproaches} erläutert. In diesem Schritt wird erläutert, welche Art LOD-System in dieser Arbeit entwickelt werden soll.

Die verschiedenen Ansätze unterscheiden sich in einigen Kriterien. Für den Einsatz in einer Webanwendungen ist die Gewichtung der Kriterien anders als für Desktopanwendungen, Spiele oder dergleichen.
So hat die Dateigrösse der Modelle bei Webanwendungen einen grösseren Einfluss auf die gefühlte Performanz beim Endanwender. Insofern ist die Auswirkung der gewählten Strategie auf die Downloadgrösse ein wichtiges Kriterium. Für diskrete LOD muss für jedes Level ein separates Modell geladen werden. Dies erhöht die Downloadgrösse merkbar. Bei kontinuierlichen LOD-Systemen muss das Modell die Unterschiede zwischen verschiedenen Levels persistieren. Mit einem für kontinuierliche LOD ausgelegten Algorithmus kann der benötigte Speicher reduziert werden, indem beim \e{Edge Collapse} nicht auf einen neuen optimalen \e{Vertex} reduziert wird, sondern von den beiden bestehenden \e{Vertices} der passendere gewählt wird. Trotzdem entsteht für jeden \e{Edge Collapse} ein zusätzlicher Speicheraufwand.
Insbesondere unter Berücksichtigung der Tatsache, dass Laufzeitperformanz essenziell für Webanwendungen ist, wurde entschieden nicht auf kontinuierliche LOD-Systeme zu setzen.
Zudem wurde entschieden kein hierarchisches LOD-System zu entwickeln, um die Integration des Tools für die Entwickler einer Webanwendung möglichst einfach zu gestalten.
Das diskrete LOD-System ist für Grafiker und Entwickler am einfachsten zu verwenden und deckt somit auch die breiteste Benutzerbasis und die meisten Anwendungsfälle ab. Aufgrund dessen, wird im Rahmen dieser Arbeit ein Algorithmus dieses Systems entwickelt und in den Entwicklungsprozess integriert.
Eine Übersicht über die verschiedenen Kriterien kann in der Tabelle \ref{table:lodSystemComparison} gefunden werden.

\begin{table}[H]
  \centering
  \begin{tabular}{||p{8cm} c c c||}
  \hline
  \textbf{Technische Auswirkungen} & DLOD & CLOD & HLOD \\
  \hline
  Auswirkung auf Downloadgrösse & mittel & mittel & mittel \\
  Auswirkung auf Laufzeitverhalten & klein & mittel & mittel \\
  Auswirkung auf \e{Scene Graph} & lokal & lokal & global \\
  "Visual Pop" vermeidbar & nein & ja & ja \\
  Integration in bestehende Arbeitsabläufe & einfach & einfach & umständlich \\
  \hline
  \textbf{Möglichkeiten} &  &  &  \\
  \hline
  Drastische Reduktion von Polygonen & ja & ja & ja \\
  Clustering möglich & nein & nein & ja \\
  \hline
  \end{tabular}
  \caption{Übersicht Merkmale der verschiedenen LOD Systeme}
  \label{table:lodSystemComparison}
\end{table}

\e{Visual Pop} bezeichnet das merkbare Austauschen zweier LOD Artefakte. Im optimalen Fall ist der Austausch zweier Artefakte nicht bemerkbar.

\subsection{Bestehende Systeme}
\label{chap:existingSolutions}

\e{Unity} sowie \e{Unreal} verfügen über umfangreiche LOD-Systeme. Eine Analyse dieser Systeme bildet somit die Grundlage für die Entwicklung eines neuen Systems.

\paragraph{Unreal}

\e{Unreal} verfügt sowohl über ein DLOD-System als auch über ein HLOD-System. Das HLOD-Tool zeichnet sich insbesondere dadurch aus, dass es die verschiedenen HLOD-Artefakte automatisch generieren kann. Dies ermöglicht es, komplexe Strukturen direkt innerhalb von \e{Unreal} zu kombinieren. \cite{unrealProxyLod}

\paragraph{Unity}

\e{Unity} verfügt über ein DLOD-System, welches es erlaubt LOD-Artefakte manuell zu definieren.
Innerhalb der \e{Unity} Community gibt es eine Bibliothek für das automatische Generieren von Artefakten, welche für LOD verwendet werden können. \cite{unityMeshSimplification}
Ansätze für ein HLOD-System sind vorhanden, wurden jedoch nicht offiziell in \e{Unity} integriert. \cite{unityAutoLod}

(TODO MB: Evtl könnte man hier noch genauer darauf eingehen, wie die Systeme funktionieren und mit Bildern zeigen.)

\section{Surface Simplification Algorithmus}
\label{chap:surfaceSimplificationAlgorithm}
Verschiedene Algorithmen wurden für die Implementierung in Betracht gezogen. Basierend auf den Erkenntnissen von Luebke wurde als Algorithmus \e{Surface Simplification using Quadric Error Metrics} gewählt \cite{luebkeAlgorithmComparison}. Der Algorithmus wurde von Garland und Heckbert 1997 definiert und basiert auf \e{Edge Collapses}. Der Ansatz überzeugt durch die Balance zwischen Performanz und Qualität \cite{surfaceSimplificationUsingQuadricErrorMetrices, surfaceSimplificationWithColorUsingQuadricErrorMetrices}.

\subsection{Grobablauf}
Der Algorithmus kann wie folgt zusammengefasst werden:

\begin{enumerate}
  \item Fehlermetrik für alle \e{Vertices} berechnen.
  \item Optionen für \e{Edge Collapse} markieren.
  \item \e{Edge Collapses} mit geringstem Fehler, solange durchführen bis gewünschte Anzahl \e{Vertices} erreicht ist.
\end{enumerate}

\subsection{Fehlermetrik}
Die Fehlermetrik dient dazu, eine Heuristik zu definieren, welche den geometrischen Unterschied zwischen dem vereinfachten und dem originalen Modell beschreibt.
Mithilfe der Fehlermetrik können anschliessend die Transformationen am Modell gemäss den kleinsten Kosten optimiert werden. Der Algorithmus funktioniert grundsätzlich unabhängig von der gewählten Fehlermetrik. So könnten verschiedene Metriken eingesetzt werden.
Die hier gewählte Metrik definiert den Fehler mithilfe einer symmetrischen $4\times 4$ Matrix. Die Matrix repräsentiert die Distanz eines \e{Vertices} zu einer zugehörigen \e{Face}. Für die Kombination mehrerer \e{Faces} können die Matrizen addiert werden.

Beim Entfernen einer \e{Edge} werden die Fehlermetriken der zugehörigen \e{Vertices} addiert.

\subsection{Referenzimplementation}
Der gewählte Algorithmus wurde bereits mehrfach implementiert. Eine Implementation basierend auf einer formatunabhängigen Datenstruktur ist jedoch nicht vorhanden. Eine oft referenzierte Implementation ist diejenige von Sven Forstmann, welche einige Optimierungen am ursprünglichen Algorithmus vornimmt, um optimale Performanz zu erhalten. Diese Optimierungen eignen sich hervorragend für eine webbasierte Implementation \cite{fastQuadricMeshSimplification}.

\section{Pipeline Integration}
Die Lösung soll in eine wiederverwendbare und konfigurierbare Pipeline integriert werden.
In der Webentwicklung ist es üblich, über \fgls{CLI}{Command Line Interface – Kommandozeile} Arbeitsschritte zu automatisieren. Es soll demnach ein \e{\gls{CLI}} erstellt und auf \gls{npm} öffentlich zugänglich gemacht werden, welches nahtlos in den Entwicklungsablauf integriert werden kann. Über eine Konfiguration soll angegeben werden können, wo nach den \e{.glTF} Dateien gesucht, wohin die Output-Dateien gespeichert und in welchem Modus das Tool gestartet werden soll. Ebenso werden Algorithmus-Einstellungen in dieser Konfiguration gesetzt werden können. Es soll einfach zu bedienen, gut dokumentiert und gut gewählte Standardeinstellungen haben. Zudem soll das Tool im einmal Modus laufen können oder kontinuierlich sich ändernde Dateien neu optimieren.

\section{Nutzen LOD}
Um den Nutzen von LOD quantifizieren zu können, müssen verschiedene Aspekte berücksichtigt werden. Auf der einen Seite ist die Entwicklungszeit zu berücksichtigen, welche sich auf die Kosten der Applikation auswirkt. Auf der anderen Seite sind Aspekte wie die Downloadzeit, die visuellen Unterschiede und die Laufzeit-Performanz der Applikation wichtig, da sie die Qualität der Software beeinflussen.

\subsection{Aspekte für Nutzen}

Der Nutzen eines Systems definiert sich durch eine Vielzahl von Aspekte, welche in verschiedene Phasen aufgeteilt werden können. Die verschiedenen für die Entwicklung von 3D-Webanwendungen relevanten Aspekte werden im folgenden Abschnitt erläutert. Es werden hierbei lediglich die Relevantesten berücksichtigt.

\paragraph{Entwicklungszeit}

Die Entwicklungszeit beziffert den Aufwand für die Entwicklung einer Anwendung und kann in Arbeitsstunden beziffert werden.
Die Auswirkungen eines Systems auf die Entwicklungszeit ist entscheidend dafür, ob etwas eingesetzt werden kann. Umso grösser der Nutzen eines Systems, desto grösser kann der Einfluss auf die Entwicklungszeit sein. Manuelle Prozesse sind wo möglich zu vermeiden. Ein System kann insofern in der Praxis nur relevant sein, wenn die Entwicklungszeit im Verhältnis zum Nutzen steht. Das Ziel muss also sein, den Aufwand und somit die Entwicklungszeit für die Nutzung von LOD möglichst gering zu halten, um die Hürde der Benutzung zu minimieren.

\paragraph{Downloadzeit}

Die Downloadzeit ist grundsätzlich linear abhängig von der Dateigrösse.
Je grösser die Dateien, desto mehr Bandbreite wird verwendet. Dies hat direkte Auswirkungen auf die Zeit, bis eine Anwendung benutzt werden kann. Somit sollte die Downloadgrösse möglichst tief gehalten werden.
Dank der heutigen Bandbreiten ist dies für viele Anwendungsbereiche ein kleiner werdendes Problem. (TODO: auch nicht auf der ganzen Welt so – muss man darauf eingehen? bzw den Punkt streichen?)
Zudem können Probleme in diesem Bereich teilweise kaschiert werden, indem man das Ladeverhalten optimiert. Häufig sind nicht alle Artefakte notwendig, um eine verwendbare Anwendung darzustellen und der Rest kann phasenweise nachgeladen werden.
Ausserdem ist es auch möglich, Artefakte bereits vorzuladen.

\paragraph{Visuelle Auswirkungen}

Die negativen visuellen Auswirkungen auf eine Anwendung sollen so gering wie möglich ausfallen. Es gilt auch in den visuellen Bereichen die richtige Balance zwischen Laufzeitverhalten und möglichst hohem Realismus zu finden. Ein Beispiel für negative visuelle Auswirkungen ist die Reduktion der Auflösung - eine tiefe Auflösung wird von vielen Benutzern als störend empfunden. Im Zusammenhang mit LOD Artefakten ist insbesondere das sogenannte \e{Visual Popping} erwähnenswert.
Beim \e{Visual Popping} handelt sich um den Moment, in dem die Artefakte ausgetauscht wird und somit kurz aufblitzt. Je subtiler die Änderungen an den Artefakten und je grösser die Distanz von der Kamera ist, desto weniger ist das Austauschen bemerkbar.

\paragraph{Laufzeit}

Entscheidend ist, unabhängig von der Applikation, das Laufzeitverhalten. (TODO SS: Was genau ist "Das Laufzeitverhalten") Eine Applikation, welche eine schlechte \e{User Experience} liefert, ist inakzeptabel.
Das Laufzeitverhalten wird durch eine Vielzahl von Aspekten beeinflusst.
Angestrebt werden grundsätzlich immer 60 \fgls{FPS}{Frames per Second. Bilder pro Sekunde. Die Anzahl der Render-Durchläufe der \gls{Rendering Engine}, die sich darauf auswirkt, wie flüssig eine Applikation läuft.}. Wenn eine bestimmte Szenerie diese 60 \e{\gls{FPS}} nicht erreicht, ist ein Performanzproblem vorhanden und das menschliche Auge kann möglicherweise ein Ruckeln wahrenehmen.

\paragraph{Berücksichtigung der Aspekte}
In dieser Arbeit wird ein hoher Wert darauf gelegt, dass die Entwicklungszeit möglichst geringfügig erhöht wird. Zudem werden, wo möglich, Empfehlungen getätigt, um die Downloadgrösse möglichst marginal zu verändern. Das Minimieren der visuellen Auswirkungen kann durch visuelle Vergleiche sichergestellt werden.
Die somit primären Aspekte für die Beurteilung sind die Auswirkungen auf das Laufzeitverhalten.

Um diese Beurteilung tätigen zu können, wird im Folgenden ein Benchmark definiert, mit dem Ziel, das Laufzeitverhalten zu analysieren und somit den maximal möglichen Einfluss von LOD auf die Leistung klassifizieren zu können. Hierbei werden verschiedene Aspekte betrachtet.

\subsection{Vergleichbare Arbeiten}
Eine unabhängige Arbeit definierte 2017 einen vergleichbaren Benchmark, um die Performanz von Three.js und Babylon.js in einem spezifischen Anwendungsgebiet zu vergleichen. Der Benchmark entwickelte vergleichbare Szenerien und vergleicht die Ressourcennutzung der beiden Varianten \cite{performanceComparisonBabylonThreejs}.

\subsection{Mögliche Ansätze}

Verschiedene Varianten können gewählt werden, um den Nutzen aufzuzeigen. Die verschiedenen in Betracht gezogenen Ansätze werden in diesem Abschnitt erläutert.

\paragraph{Modellbetrachtung}
Grundsätzlich kann der Nutzen beziffert werden, indem die Kosten zum Rendering eines Modells berechnet werden. So können die Kosten der benutzten GPU Ressourcen durch das Messen des Aufwandes eines spezifischen \e{Draw Calls} (TODO SS: Was ist ein Draw call? Evtl fussnote/glossar?) beziffert werden. Die Implementation ist simpel und liefert genaue Resultate. Die Auswirkungen von Nebeneffekten können beinahe vollständig negiert werden.
Sinnvoll wäre es, das Modell in konstanter Distanz zur Kamera um zwei Achsen drehen zu lassen und die Messung so lange durchzuführen, bis eine ganze Umdrehung stattgefunden hat. Das originale Modell kann mit den verschiedenen Stufen verglichen und die Performanzverbesserung beziffert werden.
Der primäre Nachteil bei dieser Variante liegt an der Aussagekraft. Die einzige belegte Aussage ist, dass es möglich ist, Performanzeinsparungen durch Vereinfachung zu erhalten. Dies dient jedoch nicht als Beleg, dass der Einsatz von LOD-Artefakten Performanzengpässe lösen kann, da beim Einsatz von LOD weitere Aspekte berücksichtigt werden müssen. Einige Beispiele für diese Aspekte sind: erhöhte Memory Auslastung durch die zusätzlichen Artefakte oder zusätzlich erforderliche Rechenleistung für das Berechnen des zu wählenden Artefakts.

\paragraph{Szeneriebetrachtung}
Bei diesem Ansatz wird eine gegebene Szenerie möglichst geringfügig angepasst und die Auswirkungen auf die Performanz verglichen.
Wenn die unoptimierte Variante 60 \e{\gls{FPS}} erreicht oder gar übersteigt, ist eine Optimierung nicht notwendig. Somit muss eine Szenerie gewählt werden, welche diesen Wert unterschreitet.
Das Ziel ist es dann, die LOD-Artefakte in dieselbe Szenerie zu integrieren und die \e{FPS} zu steigern. Ziel der Arbeit ist es, ein System zu entwickeln, das für Szenarien unter 60 \e{\gls{FPS}} diese erhöht und im Idealfall auf ebendiesen Wert hebt.
Eine solche Szenerie soll grundsätzlich Modelle sowohl nah als auch in grosser Distanz zeigen. Ein Beispiel für eine solche Szenerie ist eine 3D-Welt, in der man sich durch offene Gelände bewegt.
Im Vergleich zur Modellbetrachtung ist die Aussagekraft eines solchen Tests ausreichend, um einen Beleg für den positiven Nutzen des Systems zu haben. Sollte es nicht möglich sein, die \e{FPS} zu steigern, so liefert der Einsatz von LOD keinen nachweisbaren Nutzen.
