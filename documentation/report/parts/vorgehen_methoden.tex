%%% Local Variables:
%%% mode: latex
%%% TeX-master: "../doc"
%%% coding: utf-8
%%% End:
% !TEX TS-program = pdflatexmk
% !TEX encoding = UTF-8 Unicode
% !TEX root = ../doc.tex
\section{Nutzen LOD}
Um den Nutzen von LOD quantifizieren zu können, wird in einer ersten Phase ein Benchmark aufgestellt.
Ziel ist es, das Laufzeitverhalten unter Einsatz eines optimierten Modells zu analysieren und somit den maximal möglichen Einfluss von LOD auf die Leistung klassifizieren zu können.
Angestrebt werden 60 \fgls{FPS}{Frames per Second. Bilder pro Sekunde. Gängige Art, die Performanz der \gls{Rendering Engine} zu messen}. Wenn die unoptimierte Variante diesen Wert erreicht oder gar übersteigt, ist eine Optimierung nicht notwendig. Erreicht die optimierte Variante diesen Wert nicht, ist die Szenerie nach wie vor zu komplex. Ziel der Arbeit ist es, einen Algorithmus zu entwickeln, der für Szenarien unter 60 \e{\gls{FPS}} diese erhöht und im Idealfall auf ebendiesen Wert hebt.

\subsection{Definition Benchmark}
In diesem Abschnitt werden die verschiedenen erwägten Optionen erläutert und die Abgrenzungen aufgezeigt.
Das Ziel des Benchmarks ist der Vergleich einer Version ohne Einsatz von LOD und einer Umgebung, welche LODs einsetzt. Um eine möglichst praxisnahe Aussage treffen zu können wird hierfür eine Demo Szenerie verwendet, welche einer echten Anwendung so nah wie möglich kommt. Somit wird gewährleistet, dass es nicht nur in der Theorie einen Nutzen für LODs gibt sondern dieser in der Praxis vorhanden ist.

\paragraph{Browser Umgebung}
Um den Umfang des Benchmarks überschaubar zu halten, wurde ausschliesslich ein Benchmark für Google Chrome entwickelt.
Google Chrome basiert auf \fgls{Chromium}{Open Source Browser-Projekt welches von Google entwickelt wird.}, dieselbe Engine, welche auch Microsoft Edge oder Opera verwenden.
Einen Benchmark basierend auf Google Chrome liefert somit auch Indizien für diese beiden Browser, auch wenn gewisse Abweichungen möglich sind.
Neben dem Marktführer Chrome sind Mozilla Firefox oder Safari von Apple ebenfalls Optionen. Jedoch wurde primär aufgrund des Marktanteils von total rund 70\% \cite{browserUsage} zugunsten von Google Chrome entschieden.
Die getroffenen Aussagen bezüglich Laufzeitverhalten behalten ihre Gültigkeit auch für andere Browser.

\paragraph{Automation}
Um die Tests durchzuführen, wird ein Testautomationstool benötigt; unter anderem der Einsatz von Selenium wurde in Erwägung gezogen.
Der Vorteil von Selenium ist insbesondere, dass der Benchmark für weitere Browser ausgeweitet werden könnte.
Da jedoch das Analysieren der GPU Daten stark vom System abhängig ist und dafür zusätzliche Komplexität notwendig wäre, wird in diesem Benchmark die im Google Chrome integrierten \e{Chrome DevTools} eingesetzt.
Selenium bietet zurzeit eine suboptimale Integration für das \e{Chrome DevTools Protocol}.
Eine unabhängige Arbeit definierte 2017 einen vergleichbaren Benchmark um die Performanz von Three.js und Babylon.js in einem spezifischen Anwendungsgebiet zu vergleichen. Der Benchmark misst die Systemdaten unabhängig von der Browser Instanz. \cite{performanceComparisonBabylonThreejs}
Um mögliche Diskrepanzen zwischen Systemen möglichst gering zu halten wurde jedoch entschieden auf die bewährte Lösung von Google Chrome zu setzen.
\fgls{Puppeteer}{Node.js Library, die eine API anbietet zum Steuern von \gls{Chromium} oder Chrome über das Chrome DevTools Protocol}, eine weitere Option für die Automation, ist eine Bibliothek, die eine vereinfachte Schnittstelle zu einer Headless Chrome Instanz bietet.
Sie wird zudem direkt von Google entwickelt und bietet somit eine stabile Grundlage zur Kommunikation mit den \e{Chrome DevTools}.

\paragraph{Profiling}
Die \e{Chrome DevTools} erlauben es, ein detailliertes Laufzeitprofil einer Applikation anzulegen.
Im Profil befinden sich Informationen zu CPU- und GPU-Auslastung aber auch generelle Informationen bzgl. der \gls{Rendering Engine} werden gesammelt.
Die Analyse dieser Daten ermöglicht es, eine Aussage zum Laufzeitverhalten einer Applikation zu tätigen.

\paragraph{Testaufbau}
Derselbe Testablauf wird sowohl für die optimierte als auch für die unoptimierte Version verwendet.
Bei einem Testablauf werden folgende Schritte durchlaufen:

\begin{enumerate}
  \item Öffne Applikation in \emph{Headless Chrome} Instanz.
  \item Warte bis Seite geladen ist.
  \item Starte \emph{Profiling}.
  \item Warte $n$ Sekunden.
  \item Stoppe \emph{Profiling}.
  \item Werte Daten aus.
\end{enumerate}

Der Test erfasst kontinuierlich die \e{\gls{FPS}}. Zudem wird aus dem \emph{Profiling} eine Kennzahl bzgl. der GPU Nutzung berechnet.

\paragraph{Aufbau Testapplikation}
Die Testapplikation stellt eine komplexe Szenerie dar. Der Betrachter fliegt während dem Ablauf kontinuierlich über die Szenerie. Dies stellt die optimalen Bedingungen für den Einsatz von LOD Artefakten dar.

\paragraph{Testumgebung}
Um während den Tests möglichst faire Bedingungen zu gewährleisten wird die Maschine zuvor wie bei anderen Benchmarks vorbereitet. Ziel ist es, Seiteneffekte zu minimieren. Für diesen Benchmark wurden deshalb die Instruktionen von \e{Tracer Bench} zur Behebung von Rauschen befolgt.
\cite{tracerBenchNoiseMitigation}

\paragraph{Analyse der Daten}
Um eine zuverlässige Aussage treffen zu können, wird der Vorgang mehrfach wiederholt. Für jeden Durchlauf wird der Median der \e{\gls{FPS}} Daten berechnet.
Anschliessend wird die Varianz der \e{\gls{FPS}} für die unoptimierten respektive optimierten Werte berechnet. Die Varianz dient als Kennzahl, um eine Signifikanz der Daten nachweisen zu können.

\subparagraph{Konfidenzintervall}
Die Signifikanz wird mithilfe eines statistischen Konfidenzintervalls nachgewiesen. Hierfür wird der Durchschnitt aller Mediane verwendet, zusätzlich wird ein Konfidenzintervall von 95\% gewählt. Somit wird gewährleistet, dass das Resultat des Benchmarks verlässlich ist und für einen Vergleich verwendet werden kann.

\section{Vergleich LOD Systeme}
Die verschiedenen LOD Systeme sowie ihre Vor- und Nachteile werden in \autoref{chap:differentLodApproaches} erläutert. In diesem Schritt wird erläutert, welche Art LOD System in dieser Arbeit eingesetzt werden soll.

Die verschiedenen Ansätze unterscheiden sich in einigen Kriterien. Für den Einsatz in einer Webanwendungen ist die Gewichtung der Kriterien anders als für Desktopanwendungen, Spiele oder dergleichen.
So hat die Dateigrösse der Modelle bei Webanwendungen einen grösseren Einfluss auf die gefühlte Performanz beim Endanwender. Insofern ist die Auswirkung der gewählten Strategie auf die Downloadgrösse ein wichtiges Kriterium. Für diskrete LOD muss für jedes Level ein separates Modell geladen werden. Dies erhöht die Downloadgrösse merkbar. Bei kontinuierlichen LOD Systemen muss das Modell die Unterschiede zwischen verschiedenen Levels persistieren. Mit einem für kontinuierliche LOD ausgelegten Algorithmus kann der benötigte Speicher reduziert werden, indem beim \e{Edge Collapse} nicht auf einen neuen optimalen \e{Vertex} reduziert wird, sondern von den beiden bestehenden \e{Vertices} der passendere gewählt wird. Trotzdem entsteht für jeden \e{Edge Collapse} ein zusätzlicher Speicheraufwand.
Insbesondere unter Berücksichtigung der Tatsache, dass Laufzeitperformanz essentiell für Webanwendungen ist, wurde entschieden nicht auf kontinuerliche LOD Systeme zu setzen.
Zudem wurde entschieden kein hierarchisches LOD System zu entwickeln, um die Integration des Tools für die Entwickler einer Webanwendung möglichst einfach zu gestalten.
Das diskrete LOD System ist für Grafiker und Entwickler die einfachste zu verwenden und deckt somit auch die breiteste Benutzerbasis und die meisten Anwendungsfälle ab. Aufgrund dessen, wird im Rahmen dieser Arbeit ein Algorithmus dieses Systems entwickelt und in den Entwicklungsprozess integriert.
Eine Übersicht über die verschiedenen Kriterien kann in Tabelle \ref{table:lodSystemComparison} gefunden werden.

\begin{table}[H]
  \centering
  \begin{tabular}{||p{8cm} c c c||} 
  \hline
  \textbf{Technische Auswirkungen} & DLOD & CLOD & HLOD \\
  \hline
  Auswirkung auf Downloadgrösse & mittel & mittel & mittel \\ 
  Auswirkung auf Laufzeitverhalten & klein & mittel & mittel \\
  Auswirkung auf \e{Scene Graph} & lokal & lokal & global \\
  "Visual Pop" vermeidbar & nein & ja & ja \\
  Integration in bestehende Arbeitsabläufe & einfach & einfach & umständlich \\
  \hline
  \textbf{Möglichkeiten} &  &  &  \\
  \hline
  Drastische Reduktion von Polygonen & ja & ja & ja \\
  Clustering möglich & nein & nein & ja \\
  \hline
  \end{tabular}
  \caption{Übersicht Merkmale der verschiedenen LOD Systeme}
  \label{table:lodSystemComparison}
\end{table}

\section{Surface Simplification Algorithmus}
Die verschiedenen Klassen von Algorithmen werden in \autoref{chap:lodAlgorithmComparison} kurz erläutert.

Verschiedene Algorithmen wurden für die Implementierung in Betracht gezogen. Die wohl weitverbreiteste Art ist der von Garland und Heckbert 1997 definierte \e{Surface Simplification using Quadric Error Metrics} welcher auf \e{Edge Collapses} basiert.
\cite{surfaceSimplificationUsingQuadricErrorMetrices}

Der Algorithmus kann wie folgt zusammengefasst werden:

\begin{enumerate}
  \item Fehler Metrik für alle Vertices berechnen.
  \item Optionen für \e{Edge Collapse} markieren.
  \item Optionen mit geringstem Fehler optimieren.
\end{enumerate}

\subsection{Fehler Metrik}
Das Ziel der Fehler Metrik ist es eine Heuristik zu definieren, welche den geometrischen Unterschied zwischen dem vereinfachten und dem originalen Modell beschreibt.
Mithilfe der Fehler Metrik können anschliessend die Transformationen am Modell gemäss den kleinsten Kosten optimiert werden. Der Algorithmus funktioniert grundsätzlich unabhängig von der gewählten Fehler Metrik. So könnten verschiedene Metriken eingesetzt werden.
Die hier gewählte Metrik definiert den Fehler mithilfe einer symmetrischen 4×4 Matrix. Die Matrix repräsentiert die Distanz eines \e{Vertices} zu einer zugehörigen \e{Face}. Für die Kombination mehrerer \e{Faces} können die Matrizen addiert werden.

Beim Entfernen einer \e{Edge} werden die Fehler Metriken der zugehörigen \e{Vertices} addiert.

\section{Pipeline Integration}
Die Lösung soll in eine wiederverwendbare und konfigurierbare Pipeline integriert werden.
In der Webentwicklung ist es üblich, über \fgls{CLI}{Command Line Interface – Kommandozeile} Arbeitsschritte zu automatisieren. Es soll demnach ein \e{\gls{CLI}} erstellt und auf \fgls{NPM}{Node Package Manager} öffentlich zugänglich gemacht werden, welches nahtlos in den Entwicklungsablauf integriert werden kann. Über eine Konfiguration soll angegeben werden können, wo nach den \e{.glTF} Dateien gesucht, wie die Output-Dateien benennt und in welchem Modus das Tool gestartet werden soll. Ebenso werden Algorithmus-Einstellungen in dieser Konfiguration gesetzt werden können. Es soll einfach zu bedienen, gut dokumentiert und gut gewählte Standardeinstellungen haben. Zudem soll das Tool im einmal Modus laufen können oder kontinuierlich sich ändernde Dateien neu optimieren.

\section{Automatische Generierung von Detail Levels}
Damit für den Endbenutzer die Konfiguration übersichtlich bleibt, soll der Einsatz von zum Beispiel heuristischen Methoden hilfreiche Basiskonfigurationen liefern.
Zum einen geht es hier um das Festlegen der Thresholds, aber auch das Definieren des Dezimierungs- / Vereinfachungsfaktors spielt eine wichtige Rolle.

\section{Levelwahl während Laufzeit}
Die optimale Levelwahl kann stark von der Laufzeitumgebung abhängig sein. So sollte zum Beispiel auf mobilen Geräten früher ein einfacheres Modell gezeigt werden als auf leistungsstarken Geräten.
Deshalb eignet sich eine Wahl der Levels zur Laufzeit, um ein optimales Erlebnis auf allen Geräten zu ermöglichen.
Nebst dem \e{\gls{CLI}} soll folglich auch ein Javascript-Modul bereitgestellt werden, welches das Laden von glTF Dateien übernimmt und basierend auf der erkannten Performanz das beste Level of Detail einspielen.
