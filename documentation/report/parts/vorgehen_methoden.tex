%%% Local Variables:
%%% mode: latex
%%% TeX-master: "../doc"
%%% coding: utf-8
%%% End:
% !TEX TS-program = pdflatexmk
% !TEX encoding = UTF-8 Unicode
% !TEX root = ../doc.tex
\section{Nutzen LOD}
Um den Nutzen von LOD quantifizieren zu können, wird in einer ersten Phase ein Benchmark aufgestellt.
Ziel ist es, das Laufzeitverhalten unter Einsatz eines optimierten Modelles zu analysieren und somit den maximal möglichen Einfluss von LOD auf die Leistung klassifizieren zu können.

\section{Vergleich LOD Systeme}
Die unterschiedlichen LOD Systeme bieten allesamt ihre Vor- und Nachteile. In diesem Schritt wird erläutert, welche Art LOD System in dieser Arbeit eingesetzt werden soll.

\section{Vergleich LOD Algorithmen}
Abhängig vom LOD System wird ein passender Algorithmus ausgesucht.

\section{Pipeline Integration}
Die Lösung soll in eine wiederverwendbare und konfigurierbare Pipeline integriert werden.

\section{Automatische Generierurng von Detail Levels}
Damit für den Endbenutzer die Konfiguration übersichtlich bleibt, soll der Einsatz von z.B. heuristischen Methoden hilfreiche Basiskonfigurationen liefern.
Zum einen geht es hier um das festlegen der Thresholds, aber auch das definieren des Dezimierungs / Vereinfachungsfaktor spielt eine wichtige Rolle.

\section{Levelwahl während Laufzeit}
Die optimale Levelwahl kann stark von der Laufzeitumgebung abhängig sein. So sollte z.B. auf mobilen Geräten früher ein einfacheres Modell gezeigt werden als auf leistungsstarken Geräten.
Deshalb eignet sich eine Wahl der Levels zur Laufzeit, um ein optimales Erlebnis auf allen Geräten zu ermöglichen.
