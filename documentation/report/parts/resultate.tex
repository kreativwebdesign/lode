%%% Local Variables:
%%% mode: latex
%%% TeX-master: "../doc"
%%% coding: utf-8
%%% End:
% !TEX TS-program = pdflatexmk
% !TEX encoding = UTF-8 Unicode
% !TEX root = ../doc.tex
Das Resultat besteht wie beschrieben aus verschiedenen Artefakten. In diesem Abschnitt wird konkret auf die Resultate der verschiedenen Schritte eingegangen und der Zusammenhang hergestellt.

\section{Benchmark}

Der Benchmark dient als Mittel um den Nutzen von LOD in Webanwendungen quantifizieren zu können. Die dafür zur Hand genommene Masseinheit ist \gls{FPS}.

\section{LOD Pipeline}

Der Kern der Arbeit stellt die LOD Pipeline, auch \e{lode} genannt, dar. Dieses Tool liefert eine für das Web optimierte Möglichkeit LOD Artefakte für eine Vielzahl von Anwendungsgebieten zu generieren.
