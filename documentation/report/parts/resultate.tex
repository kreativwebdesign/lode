%%% Local Variables:
%%% mode: latex
%%% TeX-master: "../doc"
%%% coding: utf-8
%%% End:
% !TEX TS-program = pdflatexmk
% !TEX encoding = UTF-8 Unicode
% !TEX root = ../doc.tex
Das Resultat besteht wie beschrieben aus verschiedenen Artefakten. In diesem Abschnitt wird konkret auf die Resultate der verschiedenen Schritte eingegangen und der Zusammenhang hergestellt.

\section{Benchmark}

Der Benchmark dient als Mittel um den Nutzen von LOD in Webanwendungen quantifizieren zu können. Die dafür zur Hand genommene Masseinheit ist \gls{FPS}.

\section{LOD Pipeline}

Der Kern der Arbeit stellt die LOD Pipeline, auch \e{lode} genannt, dar. Dieses Tool liefert eine für das Web optimierte Möglichkeit LOD Artefakte für eine Vielzahl von Anwendungsgebieten zu generieren.

\subsection{Workflow}

Der übliche Ablauf für das generieren von LOD Artefakten ist wie folgt:
Für ein gegebenes Modell werden innerhalb vom Modellierungstool wie z.B. Blender bestimmte LOD Stufen von Hand generiert. Anschliessend werden die verschiedenen Stufen exportiert und manuell in die Applikation integriert. Fine Tuning erfordert so sowohl Anpassungen an den Modellen als auch im Code.

\subsection{lode Pipeline}

Das Ziel von \e{lode} ist es für ein möglichst breites Spektrum von Anwendungsfällen eine einfache Lösung anzubieten und somit die Hemmschwelle für den Einsatz von LOD Artfakten zu reduzieren.

Deshalb setzt \e{lode} konsequent auf modern Entwicklungsprozesse, manuelle Schritte sollen auf ein Minimum reduziert werden.
