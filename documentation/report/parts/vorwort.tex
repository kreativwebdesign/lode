%%% Local Variables:
%%% mode: latex
%%% TeX-master: "../doc"
%%% coding: utf-8
%%% End:
% !TEX TS-program = pdflatexmk
% !TEX encoding = UTF-8 Unicode
% !TEX root = ../doc.tex
Die Autoren der Arbeit sind seit mehreren Jahren in der Webentwicklung tätig. Der rapide Fortschritt in der Webentwicklung ist insbesondere dem Engagement der vielen Freiwilligen zu verdanken.
Insbesondere für 3D-Visualisierungen sind in den vergangenen Jahren viele neue Möglichkeiten entstanden. Aufgrund des noch jungen Alters sind die Tools noch nicht auf dem Level der anderen Plattformen. Diese Arbeit stellt insofern einen Versuch dar, etwas an die Community zurückzugeben und die Arbeit im Web im Bezug auf 3D-Applikationen zu erweitern und zu erleichtern.
Die Performanz in Webapplikationen hinkt derer nativer Applikationen hinterher. Dies bedeutet, dass jede mögliche Optimierung in dieser Hinsicht sehr wertvoll ist.
Eine Möglichkeit stellt die Vereinfachung von komplexen 3D-Modellen, welche das Original ersetzen können, wenn nicht jedes Detail wichtig ist, dar.
Ein Ursprungsmodell kann in mehreren Stufen vereinfacht werden und zum Beispiel je nach Entfernung immer weiter vereinfacht dargestellt werden, um die Performanz zu verbessern.
Zudem kann ein ultra hochauflösendes Modell für das Web vereinfacht werden, um die Downloadgrösse der Artefakten zu schmälern.
Diese sogenannten \e{Level of Details} können statisch mithilfe eines mathematischen Algorithmus' erzeugt werden.
Ziel der Arbeit ist es, ein Tool zu entwickeln, welches solche Artefakte generiert und sich nahtlos in den Arbeitsablauf von Entwicklern integriert und einfach zu benutzen ist.
