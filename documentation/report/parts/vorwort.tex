%%% Local Variables:
%%% mode: latex
%%% TeX-master: "../doc"
%%% coding: utf-8
%%% End:
% !TEX TS-program = pdflatexmk
% !TEX encoding = UTF-8 Unicode
% !TEX root = ../doc.tex
Die Autoren der Arbeit sind seit mehreren Jahren in der Webentwicklung tätig. Der rapide Fortschritt in der Webentwicklung ist insbesondere dem Engagement der vielen Freiwilligen zu verdanken.
Insbesondere für 3D-Visualisierungen sind in den vergangenen Jahren viele neue Möglichkeiten entstanden. Die Autoren teilen die Begeisterung für 3D-Visualisierungen. Aufgrund des noch jungen Alters sind die Tools noch nicht auf dem Level der anderen Plattformen. Diese Arbeit stellt insofern einen Versuch dar, etwas an die Community zurückzugeben und die Arbeit im Web im Bezug auf 3D-Applikationen zu erweitern und zu erleichtern.

Die Performanz in Webapplikationen hinkt derer nativer Applikationen hinterher. Dies bedeutet, dass jede mögliche Optimierung in dieser Hinsicht sehr wertvoll ist. Die sogenannten \e{Level of Details} sind eine Methode um die Performanz zu verbessern. Es gibt bereits Ansätze, diese sind jedoch noch nicht ausreichend ausgereift um in einer produktiven Applikation eingesetzt werden können.

Insbesondere die Anzahl der manuellen Arbeitsschritte, welche bei der Entwicklung von 3D-Applikationen notwendig sind, sind den Autoren der Arbeit ein Dorn im Auge. Ziel der Arbeit ist es somit Prozesse, welche sich in anderen Teilen der Webentwicklung bewährt haben auch für die Entwicklung von 3D-Applikationen einsetzen zu können und den manuellen Aufwand möglichst gering zu halten.

Ein besonderes Dankeschön geht an unseren Hauptbetreuer Gerrit Burkert, welcher uns während des Verfassens der Arbeit unterstützt hat.