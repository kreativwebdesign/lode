%%% Local Variables:
%%% mode: latex
%%% TeX-master: "../doc"
%%% coding: utf-8
%%% End:
% !TEX TS-program = pdflatexmk
% !TEX encoding = UTF-8 Unicode
% !TEX root = ../doc.tex
Das entwickelte Tool, \e{lode}, liefert einen ersten Schritt in die Richtung einer LOD Pipeline, welche für das Web optimiert wurde. Das Tool erreicht das Ziel für eine Vielzahl von Anwendungsgebieten einsetzbar zu sein. Wichtig ist es jedoch festzuhalten, dass es nicht für alle Anwendungen sinnvoll ist. Zum einen soll die Pipeline weiter entwickelt werden und in ihren Kernkompetenzen gestärkt werden, zum anderen gibt es jedoch auch spezifische Gebiete wie zum Beispiel Terrain für welche \e{lode} ungeeignet ist.

\section{Mögliche Erweiterungen}
Die Möglichkeiten für Erweiterungen zu \e{lode} sind sehr umfangreich. In diesem Abschnitt wird auf einige Möglichkeiten eingegangen.

\paragraph{glTF LOD Format}
Es gibt einen Proposal für die Definition von LODs innerhalb von glTF zu definieren. \cite{glTFExtensionLOD}
Zur Zeit hat Three.js jedoch noch keinen Support für diese Extension. Auch Babylon.js unterstützt lediglich das progressive Laden von Modellen. \cite{babylonProgressiveLoading}