%%% Local Variables:
%%% mode: latex
%%% TeX-master: "../doc"
%%% coding: utf-8
%%% End:
% !TEX TS-program = pdflatexmk
% !TEX encoding = UTF-8 Unicode
% !TEX root = ../doc.tex

\section{Kontext}
In der Arbeit wird der Fokus auf die Web-Plattform gelegt. Die meisten Grundlagen sind jedoch auch unabhängig davon zutreffend.
Für die Vereinfachung wird im Folgenden der Begriff "Das Web" für die Plattform von verschiedenen Web Technologien genutzt. Dies beinhaltet insbesondere vom W3C veröffentlichte Spezifikationen.

\section{Ausgangslage}
Dank der Rechenleistung auf modernen Geräten ist es möglich anspruchsvolle 3D Visualisierungen auf diversen Geräten anzuwenden. Da diese Applikationen sehr rechenintensiv sind und gerade mobile Geräte in ihrer Rechenleistung beschränkt sind, ist Performance Optimierung im 3D-Rendering unabdinglich. Insbesondere die Komplexität der Modelle hat einen signifikanten Einfluss auf die Leistung.
Eine Möglichkeit zur Optimierung ist das Anzeigen von vereinfachten Modellen ab bestimmten Distanzen zum Betrachter. So kann z.B. ein Modell in grosser Distanz vereinfacht dargestellt werden, solange bei genauer Betrachtung mehr Details sichtbar werden.
In diversen \fglspl{Rendering Engine}{Teilprogramm, das zuständig für die Darstellung von Grafiken ist} gibt es deshalb Möglichkeiten für das Verwenden von sogenannten Level Of Details (LOD) Artefakten.
Für \fgls{Game Engines}{Framework, das für den Spielverlauf und dessen Darstellung verantwortlich ist} wie Unreal oder Unity gibt es bewährte Möglichkeiten, um den Einsatz von LOD Artefakten zu vereinfachen. Zurzeit gibt es im Web Bereich keine weit verbreitete Möglichkeit für das Generieren von LOD Artefakten.

\subsection{3D-Rendering im Web}
3D-Visualisierungen werden in vielen Branchen verwendet.
Virtual Reality, CAD Anwendungen oder Computerspiele sind bekannte Anwendungsgebiete, welche 3D Visualisierungen einsetzen.
Dank leistungsstärkeren Geräten sind in vielen weiteren Bereichen realere Visualisierungen möglich.

Anwendungen können auf spezifische Hardware wie zum Beispiel die Spielkonsolen PlayStation oder Xbox ausgerichtet sein. Dies hat den Nachteil, dass die Verteilung der Software aufwändiger ist, da spezifische Hardware notwendig ist. Für gewisse Anwendungsgebiete ist dies kein Problem.

Für hardwareunabhängige Anwendungen eignet sich das Web hervorragend. Viele Benutzer haben Zugang zu einem Desktop, Tablet oder Mobiltelefon.
Somit ermöglicht das Webanwendungen mit weniger Aufwand einem grossen Zielpublikum zugänglich zu machen.
Seit einigen Jahren ist es auch möglich, 3D Visualisierungen im Web zu realisieren.
Als Basis dafür dient meist das von der Khronos Group entwickelte WebGL, das von allen modernen Browsern unterstützt wird. WebGL ist eine low-level JavaScript API für 3D Visualisierungen. \cite{webGl1Spec}
Alternativ zu WebGL wird zurzeit ein weiterer Standard entwickelt: WebGPU. Dieser ist zur Zeit des Schreibens noch in Entwicklung und wird deshalb nicht weiter berücksichtigt, auch wenn ein grosses Potenzial vorhanden ist. \cite{webGPUCharter}

Die Unabhängigkeit der Hardware bedeutet jedoch auch, dass Optimierung der Performanz in Webanwendungen unabdinglich ist, um allen Benutzern ein optimales Erlebnis zu ermöglichen.
Im Vergleich zu fixen Hardware Anwendungen ist es realistisch, dass eine Webanwendung sowohl auf einem leistungsfähigen Desktop Computer als auch auf einem günstigen Mobilgerät verwendet wird.

Zudem ist WebGL eine junge Technologie und wurde erst 2011 veröffentlicht – verglichen mit dem initialen Release Date von \fgls{OpenGL}{Spezifikation einer Programmierschnittstelle zur Entwicklung von 2D und 3D-Grafikanwendungen} welches im Jahre 1992 publiziert wurde. \cite{webGl1Spec,openGlSpec}
Nicht nur das Alter, sondern auch die Natur der Web Plattform haben dazu beigetragen, dass WebGL ein langsames Wachstum verspürt hat. Um einen Webstandard wie WebGL einsetzen zu können müssen alle grossen Browser die Spezifikation implementieren. Ansonsten kann der grosse Vorteil des Webs – einfache Verteilung an alle Benutzer – nicht in vollem Umfang genutzt werden. So hat zum Beispiel Internet Explorer 10 keinen Support und es wurde erstmals Ende 2013 möglich im Internet Explorer 11 3D Anwendungen für ein breites Publikum zu entwickeln.

\subsection{JavaScript Bibliotheken}
Für eine einfachere Arbeit mit WebGL gibt es verschiedene JavaScript Bibliotheken, welche eine Abstrahierungsschicht einfügen. Die bekanntesten werden hier kurz erläutert.

\paragraph{Three.js}
Ist die wohl weitverbreiteste Bibliothek für 3D-Rendering im Web. Die Community hinter Three.js ist aktiv und das offene Produkt wird somit konstant weiterentwickelt. Insbesondere die Erweiterungen für \e{React}, eine populäre Bibliothek für die Entwicklung von Frontend Applikationen, zeigen den Innovationsdrang.

\paragraph{Babylon.js}
Ein weiterer Kandidat ist Babylon.js, eine offene Bibliothek, welche die Entwicklung von 3D-Applikationen vereinfacht. Babylon.js zeichnet sich vor allem durch ein breites Featureset aus. 

\paragraph{PlayCanvas}
PlayCanvas ist eine offene 3D Engine, welche auch einen proprietären Cloud Service anbietet.

\paragraph{Unity}
Unity, welches vor allem für die Entwicklung von Mobile Applikationen bekannt ist, bietet seit längerer Zeit die Möglichkeit Projekte für das Web zu exportieren.

\subsection{Stand der Technik}

Wie erwähnt gibt es bereits umfangreiche LOD Systeme für komplexe Anwendungsgebiete in anderen Umgebungen. Diese werden in der Sektion \autoref{chap:existingSolutions} detaillierter erläutert.
Die erläuterten Bibliotheken bieten Funktionen für das Laden von LOD Artefakten. Teilweise gibt es die Möglichkeit Vereinfachungen im Browser zu generieren. Das Generieren von LOD Artefakten zur Laufzeit ist jedoch für Webapplikationen nicht geeignet da dies signifikante Auswirkungen auf das Laufzeitverhalten hat.
So dauert das Optimieren eines komplexen Modells wie zum Beispiel bei Babylon.js demonstriert auch auf rechnungsstarken Geräten mehrere Sekunden \cite{babylonAutoLod}.

Desweiteren sind die Systeme nicht kompatibel mit anderen Bibliotheken. So kann die Auto LOD Lösung von Babylon.js nicht in Three.js verwendet werden, obwohl die Problemstellung dies grundsätzlich erlauben würde.

\section{Zielsetzung}
Ziel der Arbeit ist es, ein Tool zu entwickeln, das den Umgang mit LOD Artefakten im Web vereinfacht. Hierfür muss vorab der Beleg erbracht werden, dass das Laufzeitverhalten der Applikation mit LOD Artfakten verbessert werden kann. Des Weiteren muss ein Algorithmus entwickelt werden, welcher es erlaubt Modelle dratisch zu vereinfachen ohne die grobe geometrische Form zu verlieren. Im Anschluss muss das Tool zur Verfügung gestellt werden, sodass es in der Praxis eingesetzt werden kann. Hierfür muss insbesondere berücksichtigt werden, dass die Artefakte nicht innerhalb des Browsers generiert werden sollen. Zudem soll der Einsatz des Tools einen möglichst geringen Zusatzaufwand bedeuten und für ein breites Spektrum an Modellen angewendet werden können. Desweiteren soll das Tool soweit erweiterbar sein, dass es für verschiedene Bibliotheken eingesetzt werden kann ohne den Kern neu entwickeln zu müssen.
