%%% Local Variables:
%%% mode: latex
%%% TeX-master: "../doc"
%%% coding: utf-8
%%% End:
% !TEX TS-program = pdflatexmk
% !TEX encoding = UTF-8 Unicode
% !TEX root = ../doc.tex
\section{Ausgangslage}

Performance Optimierung ist im 3D Rendering unabdinglich. Die Komplexität der Modelle hat dabei einen signifikanten Einfluss auf die Leistung.
Eine Möglichkeit zur Optimierung ist das generieren und verwenden von vereinfachten Modellen. In diversen Rendering Engines gibt es deshalb Möglichkeiten für das verwenden von Level Of Details (LOD) Artefakten. Dabei werden abhängig von Parametern vereinfachte Varianten desselben Modelles verwendet. So kann z.B. ein Objekt in grosser Distanz vereinfacht dargestellt werden, ohne merkbare Auswirkungen auf die Qualität zu haben.
Für Engines wie Unreal oder Unity gibt es Möglichkeiten um den Einsatz von LOD Artefakten zu vereinfachen. Im Web Bereich gibt es zur Zeit keine weit verbreitete Möglichkeit.

\section{Zielsetzung}
Ziel der Arbeit ist es ein Tool zu entwickeln das den Umgang mit LOD Artefakten im Web vereinfacht.

\subsection{Subsection}
\lipsum[1] \cite{quelle1}

\begin{figure}[H]
\centering
\includegraphics[width=0.4\columnwidth]{de-zhaw-init-rgb}
\caption{Bildli}
\label{fig:bildli1}
\end{figure}


\subsubsection{SubSubSection}
\lipsum[1]
\begin{table}[H]
\centering
\caption{Eine Tabelle}
\label{tab:my-table}
\begin{tabular}{|l|l|l|}
\hline
\textbf{A} & \textbf{B} & \textbf{C} \\ \hline
1          & 2          & 3          \\ \hline
4          & 5          & 6          \\ \hline
\end{tabular}
\end{table}

\paragraph{Paragraph}
\lipsum[1]

 


