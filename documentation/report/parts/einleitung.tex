%%% Local Variables:
%%% mode: latex
%%% TeX-master: "../doc"
%%% coding: utf-8
%%% End:
% !TEX TS-program = pdflatexmk
% !TEX encoding = UTF-8 Unicode
% !TEX root = ../doc.tex
\section{Ausgangslage}
3D-Applikation sind immer häufiger im Einsatz und werden auf diversen Geräten verwendet. Da diese Applikationen sehr rechenintensiv sind und gerade mobile Geräte in ihrer Rechenleistung beschränkt sind, ist Performance Optimierung im 3D-Rendering unabdinglich. Die Komplexität der Modelle hat dabei einen signifikanten Einfluss auf die Leistung.
Eine Möglichkeit zur Optimierung ist das Generieren und Verwenden von vereinfachten Modellen. In diversen \fgls{Rendering Engines}{Teilprogramm, das zuständig für die Darstellung von Grafiken ist} gibt es deshalb Möglichkeiten für das Verwenden von Level Of Details (LOD) Artefakten. Dabei werden abhängig von Parametern vereinfachte Varianten desselben Modelles verwendet. So kann z.B. ein Objekt in grosser Distanz vereinfacht dargestellt werden, ohne merkbare Auswirkungen auf die Qualität zu haben.
Für \fgls{Game Engines}{Framework, das für den Spielverlauf und dessen Darstellung verantwortlich ist.} wie Unreal oder Unity gibt es Möglichkeiten, um den Einsatz von LOD Artefakten zu vereinfachen. Im Web Bereich gibt es zur Zeit keine weit verbreitete Möglichkeit.

\section{Zielsetzung}
Ziel der Arbeit ist es, ein Tool zu entwickeln, das den Umgang mit LOD Artefakten im Web vereinfacht.
