%%% Local Variables:
%%% mode: latex
%%% TeX-master: "../doc"
%%% coding: utf-8
%%% End:
% !TEX TS-program = pdflatexmk
% !TEX encoding = UTF-8 Unicode
% !TEX root = ../doc.tex
\section{Ausgangslage}
Dank der Rechenleistung auf modernen Geräten ist es möglich anspruchsvolle 3D Visualisierungen auf diversen Geräten anzuwenden. Da diese Applikationen sehr rechenintensiv sind und gerade mobile Geräte in ihrer Rechenleistung beschränkt sind, ist Performance Optimierung im 3D-Rendering unabdinglich. Insbesondere die Komplexität der Modelle hat einen signifikanten Einfluss auf die Leistung.
Eine Möglichkeit zur Optimierung ist das Anzeigen von vereinfachten Modellen ab bestimmten Distanzen zum Betrachter. So kann z.B. ein Modell in grosser Distanz vereinfacht dargestellt werden, solange bei genauer Betrachtung mehr Details sichtbar werden.
In diversen \fglspl{Rendering Engine}{Teilprogramm, das zuständig für die Darstellung von Grafiken ist} gibt es deshalb Möglichkeiten für das Verwenden von sogenannten Level Of Details (LOD) Artefakten.
Für \fgls{Game Engines}{Framework, das für den Spielverlauf und dessen Darstellung verantwortlich ist} wie Unreal oder Unity gibt es bewährte Möglichkeiten, um den Einsatz von LOD Artefakten zu vereinfachen. Zurzeit gibt es im Web Bereich keine weit verbreitete Möglichkeit für das Generieren von LOD Artefakten.

\section{Zielsetzung}
Ziel der Arbeit ist es, ein Tool zu entwickeln, das den Umgang mit LOD Artefakten im Web vereinfacht. Hierfür muss vorab der Beleg erbracht werden, dass das Laufzeitverhalten der Applikation mit LOD Artfakten verbessert werden kann. Des Weiteren muss ein Algorithmus entwickelt werden, welcher es erlaubt Modelle zu vereinfachen ohne die grobe geometrische Form zu verlieren. Im Anschluss muss das Tool zur Verfügung gestellt werden, sodass es in der Praxis eingesetzt werden kann. Diese Zielsetzung definiert somit die verschiedenen Schritte der Arbeit.
