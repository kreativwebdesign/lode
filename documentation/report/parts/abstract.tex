%%% Local Variables:
%%% mode: latex
%%% TeX-master: "../doc"
%%% coding: utf-8
%%% End:
% !TEX TS-program = pdflatexmk
% !TEX encoding = UTF-8 Unicode
% !TEX root = ../doc.tex

The web platform allows 3D visualizations to be made available to a wide audience. Due to the large number of devices on which such an application can be used, performance optimization is indispensable. In this work, an option to improve runtime performance is shown by using multiple levels of detail (\e{LOD}) of a model which are generated and displayed depending on the distance of the virtual camera. Currently, considerable manual effort is required to generate and integrate such artifacts on the web.

In this paper, different approaches for generating \e{LOD} artifacts, as well as their advantages and disadvantages, are highlighted.
Furthermore, algorithms for automated simplification of polygonal models are analyzed and the chosen algorithm is implemented.

The result is a toolset that allows to generate \e{LOD} artifacts with little manual effort and to use them in 3D web applications.
It seamlessly integrates into the developer's workflow and provides a simple interactive configuration option.
By means of a benchmark, it is proven that the use of \e{LOD} can improve the runtime performance of certain 3D web applications.
