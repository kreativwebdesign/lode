%%% Local Variables:
%%% mode: latex
%%% TeX-master: "../doc"
%%% coding: utf-8
%%% End:
% !TEX TS-program = pdflatexmk
% !TEX encoding = UTF-8 Unicode
% !TEX root = ../doc.tex
The web platform allows 3D visualizations to be made available to a wide audience. Due to the large number of devices on which such an application can be used, performance optimization is essential. In this work, an option to improve runtime performance is shown by generating multiple levels of details (LOD) of a model and displaying them depending on the distance of the virtual camera. Currently, considerable manual effort is required to generate and deploy such artifacts on the web.

\bigbreak
In this work, different approaches for generating LOD artifacts, as well as their advantages and disadvantages, are shown.
Furthermore, algorithms for the simplification of polygonal models are analyzed and the chosen algorithm is implemented.
The result is a tool that makes it possible to improve the runtime performance of 3D web applications with little manual effort.
It integrates seamlessly into the developer's work process and offers a simple interactive configuration option.

\bigbreak
The result is a tool that makes it possible to improve the runtime performance of 3D web applications with little manual effort.
It integrates seamlessly into the developer's work process and offers a simple interactive configuration option.
By means of a benchmark it is proven that the use of LOD can improve the runtime performance of certain 3D web applications.
