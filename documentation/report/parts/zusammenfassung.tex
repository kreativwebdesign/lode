%%% Local Variables:
%%% mode: latex
%%% TeX-master: "../doc"
%%% coding: utf-8
%%% End:
% !TEX TS-program = pdflatexmk
% !TEX encoding = UTF-8 Unicode
% !TEX root = ../doc.tex
Die rapide Entwicklung der Webtechnologien in den vergangenen Jahren ermöglicht es, immer komplexere Applikationen für ein breites Spektrum zugänglich zu machen. Unter anderem können 3D-Visualisierungen im Webbrowser umgesetzt werden.
3D-Applikationen haben durch die zunehmende Komplexität und Grösse erhöhte Anforderungen, welche durch das junge Alter des Entwicklungsprozesses noch mühsam sind, zu erfüllen.
Durch die Vielzahl an Geräten, auf welchen ein Webbrowser auf eine solche Applikation zugreifen kann, ist das Optimieren der Performanz unabdingbar.
Zum einen ist die Downloadgrösse essenziell, um die Applikation in möglichst vielen Situationen zugänglich zu machen. Zum Anderen soll insbesondere die Rechenleistung der Geräte effizient genutzt werden, um die Applikation auf möglichst vielen Gerätearten verwenden zu können.
Ersteres kann verbessert werden, indem ein hochauflösendes 3D-Modell vorgängig vereinfacht wird. So kann die Dateigrösse verringert und insbesondere auf mobilen Geräten Daten gespart werden.
Für das zweite Problem gibt es eine Vielzahl von Lösungsansätzen. In dieser Arbeit wird eine Option aufgezeigt, indem mehrere Detailstufen (\e{Level of Details – LOD}) eines Modells generiert und je nach Entfernung der virtuellen Kamera angezeigt werden. So muss ein Objekt, welches kaum noch sichtbar, ganz hinten im Bild angezeigt wird, nicht mehr hochauflösend sein.
Zurzeit ist das Generieren und Einsetzen solcher Artefakte im Web nur bedingt automatisiert und beträchtlicher manueller Aufwand ist erforderlich.
Ziel dieser Arbeit ist, ein solches Tool zu entwickeln und nahtlos in den Arbeitsprozess der Entwickler einzugliedern.
\bigbreak
Hierfür wurden die verschiedenen Ansätze zur Generierung von solchen LOD-Artefakten verglichen und die zugehörigen Algorithmen analysiert.
Darauf basierend wurde der gewählte Algorithmus implementiert und ein Tool entworfen, diesen in den Arbeitsprozess zu integrieren.
Mittels eigenentwickelten Benchmarks wurde bewiesen, dass der Einsatz von LOD, die Laufzeitperformanz von 3D-Webapplikationen verbessert.
\bigbreak
Das Resultat ist ein Tool, welches es erlaubt mit geringem manuellen Aufwand die Laufzeitperformanz von 3D-Webapplikation zu verbessern.
