%%% Local Variables:
%%% mode: latex
%%% TeX-master: "../doc"
%%% coding: utf-8
%%% End:
% !TEX TS-program = pdflatexmk
% !TEX encoding = UTF-8 Unicode
% !TEX root = ../doc.tex
Die rapide Entwicklung der Webtechnologien erlaubt es, 3D-Visualisierungen im Browser umzusetzen. Durch die Vielzahl an Geräten, auf welchen eine solche Applikation verwendet werden kann, ist das Optimieren der Performanz unabdingbar. In dieser Arbeit wird eine Option zur Verbesserung der Laufzeitperformanz aufgezeigt, indem mehrere Detailstufen (\e{Level of Details – LOD}) eines Modells generiert und je nach Entfernung der virtuellen Kamera angezeigt werden. So muss ein Objekt, welches kaum noch sichtbar, ganz hinten im Bild angezeigt wird, nicht mehr hochauflösend sein.
Zurzeit ist das Generieren und Einsetzen solcher Artefakte im Web nur bedingt automatisiert und beträchtlicher manueller Aufwand ist erforderlich.
\bigbreak
Hierfür werden verschiedene Ansätze zur Generierung von LOD-Artefakten verglichen und zugehörige Algorithmen analysiert.
Darauf basierend wird der gewählte Algorithmus implementiert und ein Tool entworfen, diesen in den Arbeitsprozess zu integrieren.
Mittels eines Benchmarks wird bewiesen, dass der Einsatz von LOD, die Laufzeitperformanz von spezifischen 3D-Webapplikationen verbessern kann.
\bigbreak
Das Resultat ist ein Tool, welches sich nahtlos in den Arbeitsprozess der Entwickler eingliedert und es erlaubt mit geringem manuellen Aufwand die Laufzeitperformanz von 3D-Webapplikation zu verbessern.
