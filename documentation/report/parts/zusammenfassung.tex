%%% Local Variables:
%%% mode: latex
%%% TeX-master: "../doc"
%%% coding: utf-8
%%% End:
% !TEX TS-program = pdflatexmk
% !TEX encoding = UTF-8 Unicode
% !TEX root = ../doc.tex
Die Web-Plattform erlaubt es, 3D-Visualisierungen einem breiten Publikum zugänglich zu machen. Durch die Vielzahl an Geräten, auf welchen eine solche Applikation verwendet werden kann, ist das Optimieren der Performanz unabdingbar. In dieser Arbeit wird eine Option zur Verbesserung der Laufzeitleistung aufgezeigt, indem mehrere Detailstufen, sogenannte \e{Level of Details} (\e{LOD}), eines Modells generiert und je nach Entfernung der virtuellen Kamera angezeigt werden. Zurzeit ist für das Generieren und Einsetzen solcher Artefakte im Web beträchtlicher manueller Aufwand erforderlich.
\bigbreak
In dieser Arbeit werden verschiedene Ansätze zum Einsatz von LOD-Artefakten, sowie ihre Vor- und Nachteile, aufgezeigt.
Desweiteren werden Algorithmen für die automatisierte Vereinfachung von polygonalen Modellen analysiert und der gewählte Algorithmus implementiert.
\bigbreak
Das Resultat ist ein Toolset, welches es ermöglicht, mit geringem manuellen Aufwand die Laufzeitleistung von 3D-Web-Applikation zu verbessern.
Es fügt sich nahtlos in den Arbeitsprozess der Entwickler ein und bietet eine einfache interaktive Konfigurationsmöglichkeit.
Mittels eines Benchmarks wird bewiesen, dass der Einsatz von LOD die Laufzeitleistung von gewissen 3D-Web-Applikationen verbessern kann.

\todo[inline]{toolset genauer ausführen?}
\todo[inline]{Statt Pipeline neu Toolset}
