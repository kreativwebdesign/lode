%%% Local Variables:
%%% mode: latex
%%% TeX-master: "../doc"
%%% coding: utf-8
%%% End:
% !TEX TS-program = pdflatexmk
% !TEX encoding = UTF-8 Unicode
% !TEX root = ../doc.tex
\section{Diskrete LOD (DLOD)}
Bei diskreten LOD werden für ein detailliertes Modell mehrere weniger detaillierte Modelle verwendet.
Abhängig von der Distanz zum Betrachter wird das optimale Modell gewählt.
\begin{itemize}
  \pro Simplizität: Keine Anpassungen am Scene Graphh notwendig
  \con Harte Grenzen: Veränderung des Objektes kann merkbar sein
  \con Kein Clustering möglich: Probleme bei sehr grossen oder vielen kleinen Modellen
\end{itemize}

\section{Kontinuierliche LOD (CLOD)}
Im Gegensatz zu DLOD wird bei CLOD vereinfachende Veränderungen an einem Modell gespeichert.
\begin{itemize}
  \pro Weiche Grenzen: Interpolation zwischen Auflösungen ist möglich
  \con Runtime Performance
  \con Kein Clustering möglich
\end{itemize}

\section{Hierarchische LOD (HLOD)}
Bei HLOD werden mehrere Objekte in einen Cluster gruppiert.
\begin{itemize}
  \pro Clustering möglich
\end{itemize}
